\documentclass[11pt]{article}

    \usepackage[breakable]{tcolorbox}
    \usepackage{parskip} % Stop auto-indenting (to mimic markdown behaviour)
    

    % Basic figure setup, for now with no caption control since it's done
    % automatically by Pandoc (which extracts ![](path) syntax from Markdown).
    \usepackage{graphicx}
    % Maintain compatibility with old templates. Remove in nbconvert 6.0
    \let\Oldincludegraphics\includegraphics
    % Ensure that by default, figures have no caption (until we provide a
    % proper Figure object with a Caption API and a way to capture that
    % in the conversion process - todo).
    \usepackage{caption}
    \DeclareCaptionFormat{nocaption}{}
    \captionsetup{format=nocaption,aboveskip=0pt,belowskip=0pt}

    \usepackage{float}
    \floatplacement{figure}{H} % forces figures to be placed at the correct location
    \usepackage{xcolor} % Allow colors to be defined
    \usepackage{enumerate} % Needed for markdown enumerations to work
    \usepackage{geometry} % Used to adjust the document margins
    \usepackage{amsmath} % Equations
    \usepackage{amssymb} % Equations
    \usepackage{textcomp} % defines textquotesingle
    % Hack from http://tex.stackexchange.com/a/47451/13684:
    \AtBeginDocument{%
        \def\PYZsq{\textquotesingle}% Upright quotes in Pygmentized code
    }
    \usepackage{upquote} % Upright quotes for verbatim code
    \usepackage{eurosym} % defines \euro

    \usepackage{iftex}
    \ifPDFTeX
        \usepackage[T1]{fontenc}
        \IfFileExists{alphabeta.sty}{
              \usepackage{alphabeta}
          }{
              \usepackage[mathletters]{ucs}
              \usepackage[utf8x]{inputenc}
          }
    \else
        \usepackage{fontspec}
        \usepackage{unicode-math}
    \fi

    \usepackage{fancyvrb} % verbatim replacement that allows latex
    \usepackage{grffile} % extends the file name processing of package graphics
                         % to support a larger range
    \makeatletter % fix for old versions of grffile with XeLaTeX
    \@ifpackagelater{grffile}{2019/11/01}
    {
      % Do nothing on new versions
    }
    {
      \def\Gread@@xetex#1{%
        \IfFileExists{"\Gin@base".bb}%
        {\Gread@eps{\Gin@base.bb}}%
        {\Gread@@xetex@aux#1}%
      }
    }
    \makeatother
    \usepackage[Export]{adjustbox} % Used to constrain images to a maximum size
    \adjustboxset{max size={0.9\linewidth}{0.9\paperheight}}

    % The hyperref package gives us a pdf with properly built
    % internal navigation ('pdf bookmarks' for the table of contents,
    % internal cross-reference links, web links for URLs, etc.)
    \usepackage{hyperref}
    % The default LaTeX title has an obnoxious amount of whitespace. By default,
    % titling removes some of it. It also provides customization options.
    \usepackage{titling}
    \usepackage{longtable} % longtable support required by pandoc >1.10
    \usepackage{booktabs}  % table support for pandoc > 1.12.2
    \usepackage{array}     % table support for pandoc >= 2.11.3
    \usepackage{calc}      % table minipage width calculation for pandoc >= 2.11.1
    \usepackage[inline]{enumitem} % IRkernel/repr support (it uses the enumerate* environment)
    \usepackage[normalem]{ulem} % ulem is needed to support strikethroughs (\sout)
                                % normalem makes italics be italics, not underlines
    \usepackage{mathrsfs}
    

    
    % Colors for the hyperref package
    \definecolor{urlcolor}{rgb}{0,.145,.698}
    \definecolor{linkcolor}{rgb}{.71,0.21,0.01}
    \definecolor{citecolor}{rgb}{.12,.54,.11}

    % ANSI colors
    \definecolor{ansi-black}{HTML}{3E424D}
    \definecolor{ansi-black-intense}{HTML}{282C36}
    \definecolor{ansi-red}{HTML}{E75C58}
    \definecolor{ansi-red-intense}{HTML}{B22B31}
    \definecolor{ansi-green}{HTML}{00A250}
    \definecolor{ansi-green-intense}{HTML}{007427}
    \definecolor{ansi-yellow}{HTML}{DDB62B}
    \definecolor{ansi-yellow-intense}{HTML}{B27D12}
    \definecolor{ansi-blue}{HTML}{208FFB}
    \definecolor{ansi-blue-intense}{HTML}{0065CA}
    \definecolor{ansi-magenta}{HTML}{D160C4}
    \definecolor{ansi-magenta-intense}{HTML}{A03196}
    \definecolor{ansi-cyan}{HTML}{60C6C8}
    \definecolor{ansi-cyan-intense}{HTML}{258F8F}
    \definecolor{ansi-white}{HTML}{C5C1B4}
    \definecolor{ansi-white-intense}{HTML}{A1A6B2}
    \definecolor{ansi-default-inverse-fg}{HTML}{FFFFFF}
    \definecolor{ansi-default-inverse-bg}{HTML}{000000}

    % common color for the border for error outputs.
    \definecolor{outerrorbackground}{HTML}{FFDFDF}

    % commands and environments needed by pandoc snippets
    % extracted from the output of `pandoc -s`
    \providecommand{\tightlist}{%
      \setlength{\itemsep}{0pt}\setlength{\parskip}{0pt}}
    \DefineVerbatimEnvironment{Highlighting}{Verbatim}{commandchars=\\\{\}}
    % Add ',fontsize=\small' for more characters per line
    \newenvironment{Shaded}{}{}
    \newcommand{\KeywordTok}[1]{\textcolor[rgb]{0.00,0.44,0.13}{\textbf{{#1}}}}
    \newcommand{\DataTypeTok}[1]{\textcolor[rgb]{0.56,0.13,0.00}{{#1}}}
    \newcommand{\DecValTok}[1]{\textcolor[rgb]{0.25,0.63,0.44}{{#1}}}
    \newcommand{\BaseNTok}[1]{\textcolor[rgb]{0.25,0.63,0.44}{{#1}}}
    \newcommand{\FloatTok}[1]{\textcolor[rgb]{0.25,0.63,0.44}{{#1}}}
    \newcommand{\CharTok}[1]{\textcolor[rgb]{0.25,0.44,0.63}{{#1}}}
    \newcommand{\StringTok}[1]{\textcolor[rgb]{0.25,0.44,0.63}{{#1}}}
    \newcommand{\CommentTok}[1]{\textcolor[rgb]{0.38,0.63,0.69}{\textit{{#1}}}}
    \newcommand{\OtherTok}[1]{\textcolor[rgb]{0.00,0.44,0.13}{{#1}}}
    \newcommand{\AlertTok}[1]{\textcolor[rgb]{1.00,0.00,0.00}{\textbf{{#1}}}}
    \newcommand{\FunctionTok}[1]{\textcolor[rgb]{0.02,0.16,0.49}{{#1}}}
    \newcommand{\RegionMarkerTok}[1]{{#1}}
    \newcommand{\ErrorTok}[1]{\textcolor[rgb]{1.00,0.00,0.00}{\textbf{{#1}}}}
    \newcommand{\NormalTok}[1]{{#1}}

    % Additional commands for more recent versions of Pandoc
    \newcommand{\ConstantTok}[1]{\textcolor[rgb]{0.53,0.00,0.00}{{#1}}}
    \newcommand{\SpecialCharTok}[1]{\textcolor[rgb]{0.25,0.44,0.63}{{#1}}}
    \newcommand{\VerbatimStringTok}[1]{\textcolor[rgb]{0.25,0.44,0.63}{{#1}}}
    \newcommand{\SpecialStringTok}[1]{\textcolor[rgb]{0.73,0.40,0.53}{{#1}}}
    \newcommand{\ImportTok}[1]{{#1}}
    \newcommand{\DocumentationTok}[1]{\textcolor[rgb]{0.73,0.13,0.13}{\textit{{#1}}}}
    \newcommand{\AnnotationTok}[1]{\textcolor[rgb]{0.38,0.63,0.69}{\textbf{\textit{{#1}}}}}
    \newcommand{\CommentVarTok}[1]{\textcolor[rgb]{0.38,0.63,0.69}{\textbf{\textit{{#1}}}}}
    \newcommand{\VariableTok}[1]{\textcolor[rgb]{0.10,0.09,0.49}{{#1}}}
    \newcommand{\ControlFlowTok}[1]{\textcolor[rgb]{0.00,0.44,0.13}{\textbf{{#1}}}}
    \newcommand{\OperatorTok}[1]{\textcolor[rgb]{0.40,0.40,0.40}{{#1}}}
    \newcommand{\BuiltInTok}[1]{{#1}}
    \newcommand{\ExtensionTok}[1]{{#1}}
    \newcommand{\PreprocessorTok}[1]{\textcolor[rgb]{0.74,0.48,0.00}{{#1}}}
    \newcommand{\AttributeTok}[1]{\textcolor[rgb]{0.49,0.56,0.16}{{#1}}}
    \newcommand{\InformationTok}[1]{\textcolor[rgb]{0.38,0.63,0.69}{\textbf{\textit{{#1}}}}}
    \newcommand{\WarningTok}[1]{\textcolor[rgb]{0.38,0.63,0.69}{\textbf{\textit{{#1}}}}}


    % Define a nice break command that doesn't care if a line doesn't already
    % exist.
    \def\br{\hspace*{\fill} \\* }
    % Math Jax compatibility definitions
    \def\gt{>}
    \def\lt{<}
    \let\Oldtex\TeX
    \let\Oldlatex\LaTeX
    \renewcommand{\TeX}{\textrm{\Oldtex}}
    \renewcommand{\LaTeX}{\textrm{\Oldlatex}}
    % Document parameters
    % Document title
    \title{tp1-ml-victor}
    
    
    
    
    
    
    
% Pygments definitions
\makeatletter
\def\PY@reset{\let\PY@it=\relax \let\PY@bf=\relax%
    \let\PY@ul=\relax \let\PY@tc=\relax%
    \let\PY@bc=\relax \let\PY@ff=\relax}
\def\PY@tok#1{\csname PY@tok@#1\endcsname}
\def\PY@toks#1+{\ifx\relax#1\empty\else%
    \PY@tok{#1}\expandafter\PY@toks\fi}
\def\PY@do#1{\PY@bc{\PY@tc{\PY@ul{%
    \PY@it{\PY@bf{\PY@ff{#1}}}}}}}
\def\PY#1#2{\PY@reset\PY@toks#1+\relax+\PY@do{#2}}

\@namedef{PY@tok@w}{\def\PY@tc##1{\textcolor[rgb]{0.73,0.73,0.73}{##1}}}
\@namedef{PY@tok@c}{\let\PY@it=\textit\def\PY@tc##1{\textcolor[rgb]{0.24,0.48,0.48}{##1}}}
\@namedef{PY@tok@cp}{\def\PY@tc##1{\textcolor[rgb]{0.61,0.40,0.00}{##1}}}
\@namedef{PY@tok@k}{\let\PY@bf=\textbf\def\PY@tc##1{\textcolor[rgb]{0.00,0.50,0.00}{##1}}}
\@namedef{PY@tok@kp}{\def\PY@tc##1{\textcolor[rgb]{0.00,0.50,0.00}{##1}}}
\@namedef{PY@tok@kt}{\def\PY@tc##1{\textcolor[rgb]{0.69,0.00,0.25}{##1}}}
\@namedef{PY@tok@o}{\def\PY@tc##1{\textcolor[rgb]{0.40,0.40,0.40}{##1}}}
\@namedef{PY@tok@ow}{\let\PY@bf=\textbf\def\PY@tc##1{\textcolor[rgb]{0.67,0.13,1.00}{##1}}}
\@namedef{PY@tok@nb}{\def\PY@tc##1{\textcolor[rgb]{0.00,0.50,0.00}{##1}}}
\@namedef{PY@tok@nf}{\def\PY@tc##1{\textcolor[rgb]{0.00,0.00,1.00}{##1}}}
\@namedef{PY@tok@nc}{\let\PY@bf=\textbf\def\PY@tc##1{\textcolor[rgb]{0.00,0.00,1.00}{##1}}}
\@namedef{PY@tok@nn}{\let\PY@bf=\textbf\def\PY@tc##1{\textcolor[rgb]{0.00,0.00,1.00}{##1}}}
\@namedef{PY@tok@ne}{\let\PY@bf=\textbf\def\PY@tc##1{\textcolor[rgb]{0.80,0.25,0.22}{##1}}}
\@namedef{PY@tok@nv}{\def\PY@tc##1{\textcolor[rgb]{0.10,0.09,0.49}{##1}}}
\@namedef{PY@tok@no}{\def\PY@tc##1{\textcolor[rgb]{0.53,0.00,0.00}{##1}}}
\@namedef{PY@tok@nl}{\def\PY@tc##1{\textcolor[rgb]{0.46,0.46,0.00}{##1}}}
\@namedef{PY@tok@ni}{\let\PY@bf=\textbf\def\PY@tc##1{\textcolor[rgb]{0.44,0.44,0.44}{##1}}}
\@namedef{PY@tok@na}{\def\PY@tc##1{\textcolor[rgb]{0.41,0.47,0.13}{##1}}}
\@namedef{PY@tok@nt}{\let\PY@bf=\textbf\def\PY@tc##1{\textcolor[rgb]{0.00,0.50,0.00}{##1}}}
\@namedef{PY@tok@nd}{\def\PY@tc##1{\textcolor[rgb]{0.67,0.13,1.00}{##1}}}
\@namedef{PY@tok@s}{\def\PY@tc##1{\textcolor[rgb]{0.73,0.13,0.13}{##1}}}
\@namedef{PY@tok@sd}{\let\PY@it=\textit\def\PY@tc##1{\textcolor[rgb]{0.73,0.13,0.13}{##1}}}
\@namedef{PY@tok@si}{\let\PY@bf=\textbf\def\PY@tc##1{\textcolor[rgb]{0.64,0.35,0.47}{##1}}}
\@namedef{PY@tok@se}{\let\PY@bf=\textbf\def\PY@tc##1{\textcolor[rgb]{0.67,0.36,0.12}{##1}}}
\@namedef{PY@tok@sr}{\def\PY@tc##1{\textcolor[rgb]{0.64,0.35,0.47}{##1}}}
\@namedef{PY@tok@ss}{\def\PY@tc##1{\textcolor[rgb]{0.10,0.09,0.49}{##1}}}
\@namedef{PY@tok@sx}{\def\PY@tc##1{\textcolor[rgb]{0.00,0.50,0.00}{##1}}}
\@namedef{PY@tok@m}{\def\PY@tc##1{\textcolor[rgb]{0.40,0.40,0.40}{##1}}}
\@namedef{PY@tok@gh}{\let\PY@bf=\textbf\def\PY@tc##1{\textcolor[rgb]{0.00,0.00,0.50}{##1}}}
\@namedef{PY@tok@gu}{\let\PY@bf=\textbf\def\PY@tc##1{\textcolor[rgb]{0.50,0.00,0.50}{##1}}}
\@namedef{PY@tok@gd}{\def\PY@tc##1{\textcolor[rgb]{0.63,0.00,0.00}{##1}}}
\@namedef{PY@tok@gi}{\def\PY@tc##1{\textcolor[rgb]{0.00,0.52,0.00}{##1}}}
\@namedef{PY@tok@gr}{\def\PY@tc##1{\textcolor[rgb]{0.89,0.00,0.00}{##1}}}
\@namedef{PY@tok@ge}{\let\PY@it=\textit}
\@namedef{PY@tok@gs}{\let\PY@bf=\textbf}
\@namedef{PY@tok@gp}{\let\PY@bf=\textbf\def\PY@tc##1{\textcolor[rgb]{0.00,0.00,0.50}{##1}}}
\@namedef{PY@tok@go}{\def\PY@tc##1{\textcolor[rgb]{0.44,0.44,0.44}{##1}}}
\@namedef{PY@tok@gt}{\def\PY@tc##1{\textcolor[rgb]{0.00,0.27,0.87}{##1}}}
\@namedef{PY@tok@err}{\def\PY@bc##1{{\setlength{\fboxsep}{\string -\fboxrule}\fcolorbox[rgb]{1.00,0.00,0.00}{1,1,1}{\strut ##1}}}}
\@namedef{PY@tok@kc}{\let\PY@bf=\textbf\def\PY@tc##1{\textcolor[rgb]{0.00,0.50,0.00}{##1}}}
\@namedef{PY@tok@kd}{\let\PY@bf=\textbf\def\PY@tc##1{\textcolor[rgb]{0.00,0.50,0.00}{##1}}}
\@namedef{PY@tok@kn}{\let\PY@bf=\textbf\def\PY@tc##1{\textcolor[rgb]{0.00,0.50,0.00}{##1}}}
\@namedef{PY@tok@kr}{\let\PY@bf=\textbf\def\PY@tc##1{\textcolor[rgb]{0.00,0.50,0.00}{##1}}}
\@namedef{PY@tok@bp}{\def\PY@tc##1{\textcolor[rgb]{0.00,0.50,0.00}{##1}}}
\@namedef{PY@tok@fm}{\def\PY@tc##1{\textcolor[rgb]{0.00,0.00,1.00}{##1}}}
\@namedef{PY@tok@vc}{\def\PY@tc##1{\textcolor[rgb]{0.10,0.09,0.49}{##1}}}
\@namedef{PY@tok@vg}{\def\PY@tc##1{\textcolor[rgb]{0.10,0.09,0.49}{##1}}}
\@namedef{PY@tok@vi}{\def\PY@tc##1{\textcolor[rgb]{0.10,0.09,0.49}{##1}}}
\@namedef{PY@tok@vm}{\def\PY@tc##1{\textcolor[rgb]{0.10,0.09,0.49}{##1}}}
\@namedef{PY@tok@sa}{\def\PY@tc##1{\textcolor[rgb]{0.73,0.13,0.13}{##1}}}
\@namedef{PY@tok@sb}{\def\PY@tc##1{\textcolor[rgb]{0.73,0.13,0.13}{##1}}}
\@namedef{PY@tok@sc}{\def\PY@tc##1{\textcolor[rgb]{0.73,0.13,0.13}{##1}}}
\@namedef{PY@tok@dl}{\def\PY@tc##1{\textcolor[rgb]{0.73,0.13,0.13}{##1}}}
\@namedef{PY@tok@s2}{\def\PY@tc##1{\textcolor[rgb]{0.73,0.13,0.13}{##1}}}
\@namedef{PY@tok@sh}{\def\PY@tc##1{\textcolor[rgb]{0.73,0.13,0.13}{##1}}}
\@namedef{PY@tok@s1}{\def\PY@tc##1{\textcolor[rgb]{0.73,0.13,0.13}{##1}}}
\@namedef{PY@tok@mb}{\def\PY@tc##1{\textcolor[rgb]{0.40,0.40,0.40}{##1}}}
\@namedef{PY@tok@mf}{\def\PY@tc##1{\textcolor[rgb]{0.40,0.40,0.40}{##1}}}
\@namedef{PY@tok@mh}{\def\PY@tc##1{\textcolor[rgb]{0.40,0.40,0.40}{##1}}}
\@namedef{PY@tok@mi}{\def\PY@tc##1{\textcolor[rgb]{0.40,0.40,0.40}{##1}}}
\@namedef{PY@tok@il}{\def\PY@tc##1{\textcolor[rgb]{0.40,0.40,0.40}{##1}}}
\@namedef{PY@tok@mo}{\def\PY@tc##1{\textcolor[rgb]{0.40,0.40,0.40}{##1}}}
\@namedef{PY@tok@ch}{\let\PY@it=\textit\def\PY@tc##1{\textcolor[rgb]{0.24,0.48,0.48}{##1}}}
\@namedef{PY@tok@cm}{\let\PY@it=\textit\def\PY@tc##1{\textcolor[rgb]{0.24,0.48,0.48}{##1}}}
\@namedef{PY@tok@cpf}{\let\PY@it=\textit\def\PY@tc##1{\textcolor[rgb]{0.24,0.48,0.48}{##1}}}
\@namedef{PY@tok@c1}{\let\PY@it=\textit\def\PY@tc##1{\textcolor[rgb]{0.24,0.48,0.48}{##1}}}
\@namedef{PY@tok@cs}{\let\PY@it=\textit\def\PY@tc##1{\textcolor[rgb]{0.24,0.48,0.48}{##1}}}

\def\PYZbs{\char`\\}
\def\PYZus{\char`\_}
\def\PYZob{\char`\{}
\def\PYZcb{\char`\}}
\def\PYZca{\char`\^}
\def\PYZam{\char`\&}
\def\PYZlt{\char`\<}
\def\PYZgt{\char`\>}
\def\PYZsh{\char`\#}
\def\PYZpc{\char`\%}
\def\PYZdl{\char`\$}
\def\PYZhy{\char`\-}
\def\PYZsq{\char`\'}
\def\PYZdq{\char`\"}
\def\PYZti{\char`\~}
% for compatibility with earlier versions
\def\PYZat{@}
\def\PYZlb{[}
\def\PYZrb{]}
\makeatother


    % For linebreaks inside Verbatim environment from package fancyvrb.
    \makeatletter
        \newbox\Wrappedcontinuationbox
        \newbox\Wrappedvisiblespacebox
        \newcommand*\Wrappedvisiblespace {\textcolor{red}{\textvisiblespace}}
        \newcommand*\Wrappedcontinuationsymbol {\textcolor{red}{\llap{\tiny$\m@th\hookrightarrow$}}}
        \newcommand*\Wrappedcontinuationindent {3ex }
        \newcommand*\Wrappedafterbreak {\kern\Wrappedcontinuationindent\copy\Wrappedcontinuationbox}
        % Take advantage of the already applied Pygments mark-up to insert
        % potential linebreaks for TeX processing.
        %        {, <, #, %, $, ' and ": go to next line.
        %        _, }, ^, &, >, - and ~: stay at end of broken line.
        % Use of \textquotesingle for straight quote.
        \newcommand*\Wrappedbreaksatspecials {%
            \def\PYGZus{\discretionary{\char`\_}{\Wrappedafterbreak}{\char`\_}}%
            \def\PYGZob{\discretionary{}{\Wrappedafterbreak\char`\{}{\char`\{}}%
            \def\PYGZcb{\discretionary{\char`\}}{\Wrappedafterbreak}{\char`\}}}%
            \def\PYGZca{\discretionary{\char`\^}{\Wrappedafterbreak}{\char`\^}}%
            \def\PYGZam{\discretionary{\char`\&}{\Wrappedafterbreak}{\char`\&}}%
            \def\PYGZlt{\discretionary{}{\Wrappedafterbreak\char`\<}{\char`\<}}%
            \def\PYGZgt{\discretionary{\char`\>}{\Wrappedafterbreak}{\char`\>}}%
            \def\PYGZsh{\discretionary{}{\Wrappedafterbreak\char`\#}{\char`\#}}%
            \def\PYGZpc{\discretionary{}{\Wrappedafterbreak\char`\%}{\char`\%}}%
            \def\PYGZdl{\discretionary{}{\Wrappedafterbreak\char`\$}{\char`\$}}%
            \def\PYGZhy{\discretionary{\char`\-}{\Wrappedafterbreak}{\char`\-}}%
            \def\PYGZsq{\discretionary{}{\Wrappedafterbreak\textquotesingle}{\textquotesingle}}%
            \def\PYGZdq{\discretionary{}{\Wrappedafterbreak\char`\"}{\char`\"}}%
            \def\PYGZti{\discretionary{\char`\~}{\Wrappedafterbreak}{\char`\~}}%
        }
        % Some characters . , ; ? ! / are not pygmentized.
        % This macro makes them "active" and they will insert potential linebreaks
        \newcommand*\Wrappedbreaksatpunct {%
            \lccode`\~`\.\lowercase{\def~}{\discretionary{\hbox{\char`\.}}{\Wrappedafterbreak}{\hbox{\char`\.}}}%
            \lccode`\~`\,\lowercase{\def~}{\discretionary{\hbox{\char`\,}}{\Wrappedafterbreak}{\hbox{\char`\,}}}%
            \lccode`\~`\;\lowercase{\def~}{\discretionary{\hbox{\char`\;}}{\Wrappedafterbreak}{\hbox{\char`\;}}}%
            \lccode`\~`\:\lowercase{\def~}{\discretionary{\hbox{\char`\:}}{\Wrappedafterbreak}{\hbox{\char`\:}}}%
            \lccode`\~`\?\lowercase{\def~}{\discretionary{\hbox{\char`\?}}{\Wrappedafterbreak}{\hbox{\char`\?}}}%
            \lccode`\~`\!\lowercase{\def~}{\discretionary{\hbox{\char`\!}}{\Wrappedafterbreak}{\hbox{\char`\!}}}%
            \lccode`\~`\/\lowercase{\def~}{\discretionary{\hbox{\char`\/}}{\Wrappedafterbreak}{\hbox{\char`\/}}}%
            \catcode`\.\active
            \catcode`\,\active
            \catcode`\;\active
            \catcode`\:\active
            \catcode`\?\active
            \catcode`\!\active
            \catcode`\/\active
            \lccode`\~`\~
        }
    \makeatother

    \let\OriginalVerbatim=\Verbatim
    \makeatletter
    \renewcommand{\Verbatim}[1][1]{%
        %\parskip\z@skip
        \sbox\Wrappedcontinuationbox {\Wrappedcontinuationsymbol}%
        \sbox\Wrappedvisiblespacebox {\FV@SetupFont\Wrappedvisiblespace}%
        \def\FancyVerbFormatLine ##1{\hsize\linewidth
            \vtop{\raggedright\hyphenpenalty\z@\exhyphenpenalty\z@
                \doublehyphendemerits\z@\finalhyphendemerits\z@
                \strut ##1\strut}%
        }%
        % If the linebreak is at a space, the latter will be displayed as visible
        % space at end of first line, and a continuation symbol starts next line.
        % Stretch/shrink are however usually zero for typewriter font.
        \def\FV@Space {%
            \nobreak\hskip\z@ plus\fontdimen3\font minus\fontdimen4\font
            \discretionary{\copy\Wrappedvisiblespacebox}{\Wrappedafterbreak}
            {\kern\fontdimen2\font}%
        }%

        % Allow breaks at special characters using \PYG... macros.
        \Wrappedbreaksatspecials
        % Breaks at punctuation characters . , ; ? ! and / need catcode=\active
        \OriginalVerbatim[#1,codes*=\Wrappedbreaksatpunct]%
    }
    \makeatother

    % Exact colors from NB
    \definecolor{incolor}{HTML}{303F9F}
    \definecolor{outcolor}{HTML}{D84315}
    \definecolor{cellborder}{HTML}{CFCFCF}
    \definecolor{cellbackground}{HTML}{F7F7F7}

    % prompt
    \makeatletter
    \newcommand{\boxspacing}{\kern\kvtcb@left@rule\kern\kvtcb@boxsep}
    \makeatother
    \newcommand{\prompt}[4]{
        {\ttfamily\llap{{\color{#2}[#3]:\hspace{3pt}#4}}\vspace{-\baselineskip}}
    }
    

    
    % Prevent overflowing lines due to hard-to-break entities
    \sloppy
    % Setup hyperref package
    \hypersetup{
      breaklinks=true,  % so long urls are correctly broken across lines
      colorlinks=true,
      urlcolor=urlcolor,
      linkcolor=linkcolor,
      citecolor=citecolor,
      }
    % Slightly bigger margins than the latex defaults
    
    \geometry{verbose,tmargin=1in,bmargin=1in,lmargin=1in,rmargin=1in}
    
    

\begin{document}
    
    \maketitle
    
    

    
    Importando todas as bibliotecas necessárias

    \begin{tcolorbox}[breakable, size=fbox, boxrule=1pt, pad at break*=1mm,colback=cellbackground, colframe=cellborder]
\prompt{In}{incolor}{1}{\boxspacing}
\begin{Verbatim}[commandchars=\\\{\}]
\PY{k+kn}{import} \PY{n+nn}{pickle}
\PY{k+kn}{from} \PY{n+nn}{sklearn}\PY{n+nn}{.}\PY{n+nn}{model\PYZus{}selection} \PY{k+kn}{import} \PY{n}{train\PYZus{}test\PYZus{}split}
\PY{k+kn}{from} \PY{n+nn}{keras}\PY{n+nn}{.}\PY{n+nn}{models} \PY{k+kn}{import} \PY{n}{load\PYZus{}model}
\PY{k+kn}{from} \PY{n+nn}{tensorflow}\PY{n+nn}{.}\PY{n+nn}{keras} \PY{k+kn}{import} \PY{n}{layers}
\PY{k+kn}{import} \PY{n+nn}{matplotlib}\PY{n+nn}{.}\PY{n+nn}{pyplot} \PY{k}{as} \PY{n+nn}{plt}
\PY{k+kn}{import} \PY{n+nn}{tensorflow} \PY{k}{as} \PY{n+nn}{tf}
\PY{k+kn}{import} \PY{n+nn}{numpy} \PY{k}{as} \PY{n+nn}{np}
\PY{k+kn}{import} \PY{n+nn}{os}
\PY{k+kn}{import} \PY{n+nn}{json}
\PY{k+kn}{from} \PY{n+nn}{keras}\PY{n+nn}{.}\PY{n+nn}{callbacks} \PY{k+kn}{import} \PY{n}{History}
\end{Verbatim}
\end{tcolorbox}

    \begin{Verbatim}[commandchars=\\\{\}]
2023-05-24 22:24:14.867259: I tensorflow/tsl/cuda/cudart\_stub.cc:28] Could not
find cuda drivers on your machine, GPU will not be used.
2023-05-24 22:24:15.063304: I tensorflow/tsl/cuda/cudart\_stub.cc:28] Could not
find cuda drivers on your machine, GPU will not be used.
2023-05-24 22:24:15.064542: I tensorflow/core/platform/cpu\_feature\_guard.cc:182]
This TensorFlow binary is optimized to use available CPU instructions in
performance-critical operations.
To enable the following instructions: AVX2 FMA, in other operations, rebuild
TensorFlow with the appropriate compiler flags.
2023-05-24 22:24:16.498716: W
tensorflow/compiler/tf2tensorrt/utils/py\_utils.cc:38] TF-TRT Warning: Could not
find TensorRT
    \end{Verbatim}

    Treinando os 36 modelos e salvando em disco

    \begin{tcolorbox}[breakable, size=fbox, boxrule=1pt, pad at break*=1mm,colback=cellbackground, colframe=cellborder]
\prompt{In}{incolor}{ }{\boxspacing}
\begin{Verbatim}[commandchars=\\\{\}]
\PY{k}{def} \PY{n+nf}{trainMLP}\PY{p}{(}\PY{n}{metric}\PY{p}{,} \PY{n}{epochs}\PY{p}{,} \PY{n}{hidden\PYZus{}layer\PYZus{}neuron\PYZus{}number}\PY{p}{,} \PY{n}{learning\PYZus{}rate}\PY{p}{,} \PY{n}{batch\PYZus{}size}\PY{p}{,} \PY{n}{test\PYZus{}set\PYZus{}percentage}\PY{p}{)}\PY{p}{:}
    \PY{c+c1}{\PYZsh{} definindo o modelo}
    \PY{n}{mlp} \PY{o}{=} \PY{n}{tf}\PY{o}{.}\PY{n}{keras}\PY{o}{.}\PY{n}{models}\PY{o}{.}\PY{n}{Sequential}\PY{p}{(}\PY{p}{)}
    \PY{n}{mlp}\PY{o}{.}\PY{n}{add}\PY{p}{(}\PY{n}{layers}\PY{o}{.}\PY{n}{Dense}\PY{p}{(}\PY{l+m+mi}{784}\PY{p}{,} \PY{n}{activation}\PY{o}{=}\PY{l+s+s1}{\PYZsq{}}\PY{l+s+s1}{sigmoid}\PY{l+s+s1}{\PYZsq{}}\PY{p}{,} \PY{n}{input\PYZus{}shape}\PY{o}{=}\PY{p}{(}\PY{l+m+mi}{784}\PY{p}{,}\PY{p}{)}\PY{p}{)}\PY{p}{)}
    \PY{n}{mlp}\PY{o}{.}\PY{n}{add}\PY{p}{(}\PY{n}{layers}\PY{o}{.}\PY{n}{Dense}\PY{p}{(}\PY{n}{hidden\PYZus{}layer\PYZus{}neuron\PYZus{}number}\PY{p}{,} \PY{n}{activation}\PY{o}{=}\PY{l+s+s1}{\PYZsq{}}\PY{l+s+s1}{sigmoid}\PY{l+s+s1}{\PYZsq{}}\PY{p}{)}\PY{p}{)}
    \PY{n}{mlp}\PY{o}{.}\PY{n}{add}\PY{p}{(}\PY{n}{layers}\PY{o}{.}\PY{n}{Dense}\PY{p}{(}\PY{l+m+mi}{10}\PY{p}{,} \PY{n}{activation}\PY{o}{=}\PY{l+s+s1}{\PYZsq{}}\PY{l+s+s1}{sigmoid}\PY{l+s+s1}{\PYZsq{}}\PY{p}{)}\PY{p}{)}

    \PY{c+c1}{\PYZsh{} compilando o modelo}
    \PY{n}{optimizer} \PY{o}{=} \PY{n}{tf}\PY{o}{.}\PY{n}{keras}\PY{o}{.}\PY{n}{optimizers}\PY{o}{.}\PY{n}{SGD}\PY{p}{(}\PY{n}{learning\PYZus{}rate}\PY{o}{=}\PY{n}{learning\PYZus{}rate}\PY{p}{)}
    \PY{n}{mlp}\PY{o}{.}\PY{n}{compile}\PY{p}{(}\PY{n}{optimizer}\PY{o}{=}\PY{n}{optimizer}\PY{p}{,} \PY{n}{loss}\PY{o}{=}\PY{l+s+s1}{\PYZsq{}}\PY{l+s+s1}{sparse\PYZus{}categorical\PYZus{}crossentropy}\PY{l+s+s1}{\PYZsq{}}\PY{p}{,} \PY{n}{metrics}\PY{o}{=}\PY{p}{[}\PY{n}{metric}\PY{p}{]}\PY{p}{)}

    \PY{c+c1}{\PYZsh{} lendo os dados do arquivo de texto}
    \PY{n}{data} \PY{o}{=} \PY{n}{np}\PY{o}{.}\PY{n}{loadtxt}\PY{p}{(}\PY{l+s+s1}{\PYZsq{}}\PY{l+s+s1}{data\PYZus{}tp1}\PY{l+s+s1}{\PYZsq{}}\PY{p}{,} \PY{n}{delimiter}\PY{o}{=}\PY{l+s+s1}{\PYZsq{}}\PY{l+s+s1}{,}\PY{l+s+s1}{\PYZsq{}}\PY{p}{)}
    \PY{n}{X} \PY{o}{=} \PY{n}{data}\PY{p}{[}\PY{p}{:}\PY{p}{,} \PY{l+m+mi}{1}\PY{p}{:}\PY{p}{]}
    \PY{n}{y} \PY{o}{=} \PY{n}{data}\PY{p}{[}\PY{p}{:}\PY{p}{,} \PY{l+m+mi}{0}\PY{p}{]}

    \PY{n}{history} \PY{o}{=} \PY{n}{History}\PY{p}{(}\PY{p}{)}
    \PY{n}{mlp}\PY{o}{.}\PY{n}{fit}\PY{p}{(}\PY{n}{X}\PY{p}{,} \PY{n}{y}\PY{p}{,} \PY{n}{validation\PYZus{}split}\PY{o}{=}\PY{n}{test\PYZus{}set\PYZus{}percentage}\PY{p}{,} \PY{n}{epochs}\PY{o}{=}\PY{n}{epochs}\PY{p}{,} \PY{n}{batch\PYZus{}size}\PY{o}{=}\PY{n}{batch\PYZus{}size}\PY{p}{,} \PY{n}{verbose}\PY{o}{=}\PY{l+m+mi}{1}\PY{p}{,} \PY{n}{shuffle}\PY{o}{=}\PY{k+kc}{True}\PY{p}{,} \PY{n}{workers}\PY{o}{=}\PY{l+m+mi}{8}\PY{p}{,} \PY{n}{callbacks}\PY{o}{=}\PY{p}{[}\PY{n}{history}\PY{p}{]}\PY{p}{)}
    \PY{k}{return} \PY{p}{(}\PY{n}{mlp}\PY{p}{,} \PY{n}{history}\PY{p}{)}

\PY{k}{def} \PY{n+nf}{saveMLP}\PY{p}{(}\PY{n}{base\PYZus{}folder}\PY{p}{,} \PY{n}{mlp}\PY{p}{,} \PY{n}{learning\PYZus{}rate}\PY{p}{,} \PY{n}{history}\PY{p}{)}\PY{p}{:}
    \PY{k}{if} \PY{o+ow}{not} \PY{n}{os}\PY{o}{.}\PY{n}{path}\PY{o}{.}\PY{n}{exists}\PY{p}{(}\PY{l+s+s1}{\PYZsq{}}\PY{l+s+s1}{models}\PY{l+s+s1}{\PYZsq{}}\PY{p}{)}\PY{p}{:}
        \PY{n}{os}\PY{o}{.}\PY{n}{makedirs}\PY{p}{(}\PY{l+s+s1}{\PYZsq{}}\PY{l+s+s1}{models}\PY{l+s+s1}{\PYZsq{}}\PY{p}{)}

    \PY{k}{with} \PY{n+nb}{open}\PY{p}{(}\PY{l+s+sa}{f}\PY{l+s+s2}{\PYZdq{}}\PY{l+s+si}{\PYZob{}}\PY{n}{base\PYZus{}folder}\PY{l+s+si}{\PYZcb{}}\PY{l+s+s2}{/history\PYZhy{}lr}\PY{l+s+si}{\PYZob{}}\PY{n}{learning\PYZus{}rate}\PY{l+s+si}{\PYZcb{}}\PY{l+s+s2}{.pkl}\PY{l+s+s2}{\PYZdq{}}\PY{p}{,} \PY{l+s+s1}{\PYZsq{}}\PY{l+s+s1}{wb}\PY{l+s+s1}{\PYZsq{}}\PY{p}{)} \PY{k}{as} \PY{n}{file}\PY{p}{:}
        \PY{n}{pickle}\PY{o}{.}\PY{n}{dump}\PY{p}{(}\PY{n}{history}\PY{o}{.}\PY{n}{history}\PY{p}{,} \PY{n}{file}\PY{p}{)}
        
    \PY{n}{mlp}\PY{o}{.}\PY{n}{save}\PY{p}{(}\PY{l+s+sa}{f}\PY{l+s+s2}{\PYZdq{}}\PY{l+s+si}{\PYZob{}}\PY{n}{base\PYZus{}folder}\PY{l+s+si}{\PYZcb{}}\PY{l+s+s2}{/mlp\PYZhy{}lr}\PY{l+s+si}{\PYZob{}}\PY{n}{learning\PYZus{}rate}\PY{l+s+si}{\PYZcb{}}\PY{l+s+s2}{\PYZdq{}}\PY{p}{)}

\PY{n}{params} \PY{o}{=} \PY{n}{json}\PY{o}{.}\PY{n}{load}\PY{p}{(}\PY{n+nb}{open}\PY{p}{(}\PY{l+s+s1}{\PYZsq{}}\PY{l+s+s1}{params.json}\PY{l+s+s1}{\PYZsq{}}\PY{p}{)}\PY{p}{)}

\PY{n}{test\PYZus{}set\PYZus{}percentage} \PY{o}{=} \PY{n}{params}\PY{p}{[}\PY{l+s+s1}{\PYZsq{}}\PY{l+s+s1}{test\PYZus{}set\PYZus{}percentage}\PY{l+s+s1}{\PYZsq{}}\PY{p}{]}
\PY{n}{metric} \PY{o}{=} \PY{n}{params}\PY{p}{[}\PY{l+s+s1}{\PYZsq{}}\PY{l+s+s1}{metric}\PY{l+s+s1}{\PYZsq{}}\PY{p}{]}
\PY{n}{epochs} \PY{o}{=} \PY{n}{params}\PY{p}{[}\PY{l+s+s1}{\PYZsq{}}\PY{l+s+s1}{epochs}\PY{l+s+s1}{\PYZsq{}}\PY{p}{]}

\PY{c+c1}{\PYZsh{} Ordenado de forma que as primeiras execuções sejam mais rápidas}
\PY{n}{batch\PYZus{}sizes} \PY{o}{=} \PY{p}{[}\PY{n+nb}{round}\PY{p}{(}\PY{l+m+mi}{5000} \PY{o}{*} \PY{n+nb}{round}\PY{p}{(}\PY{l+m+mi}{1} \PY{o}{\PYZhy{}} \PY{n}{params}\PY{p}{[}\PY{l+s+s1}{\PYZsq{}}\PY{l+s+s1}{test\PYZus{}set\PYZus{}percentage}\PY{l+s+s1}{\PYZsq{}}\PY{p}{]}\PY{p}{,} \PY{l+m+mi}{2}\PY{p}{)}\PY{p}{)}\PY{p}{]}
\PY{k}{for} \PY{n}{batch\PYZus{}size} \PY{o+ow}{in} \PY{n}{params}\PY{o}{.}\PY{n}{get}\PY{p}{(}\PY{l+s+s1}{\PYZsq{}}\PY{l+s+s1}{batch\PYZus{}sizes}\PY{l+s+s1}{\PYZsq{}}\PY{p}{)}\PY{p}{:}
    \PY{n}{batch\PYZus{}sizes}\PY{o}{.}\PY{n}{append}\PY{p}{(}\PY{n}{batch\PYZus{}size}\PY{p}{)}
\PY{c+c1}{\PYZsh{} batch\PYZus{}sizes = [3750, 50, 10, 1] Originalmente}

\PY{n}{hidden\PYZus{}layer\PYZus{}neuron\PYZus{}numbers} \PY{o}{=} \PY{n}{params}\PY{p}{[}\PY{l+s+s1}{\PYZsq{}}\PY{l+s+s1}{hidden\PYZus{}layer\PYZus{}neuron\PYZus{}numbers}\PY{l+s+s1}{\PYZsq{}}\PY{p}{]}
\PY{n}{learning\PYZus{}rates} \PY{o}{=} \PY{n}{params}\PY{p}{[}\PY{l+s+s1}{\PYZsq{}}\PY{l+s+s1}{learning\PYZus{}rates}\PY{l+s+s1}{\PYZsq{}}\PY{p}{]}

\PY{k}{for} \PY{n}{batch\PYZus{}size} \PY{o+ow}{in} \PY{n}{batch\PYZus{}sizes}\PY{p}{:}
    \PY{n}{batch\PYZus{}folder} \PY{o}{=} \PY{l+s+sa}{f}\PY{l+s+s2}{\PYZdq{}}\PY{l+s+s2}{models/batch\PYZhy{}size\PYZhy{}}\PY{l+s+si}{\PYZob{}}\PY{n}{batch\PYZus{}size}\PY{l+s+si}{\PYZcb{}}\PY{l+s+s2}{\PYZdq{}}
    \PY{k}{if} \PY{o+ow}{not} \PY{n}{os}\PY{o}{.}\PY{n}{path}\PY{o}{.}\PY{n}{exists}\PY{p}{(}\PY{n}{batch\PYZus{}folder}\PY{p}{)}\PY{p}{:}
        \PY{n}{os}\PY{o}{.}\PY{n}{makedirs}\PY{p}{(}\PY{n}{batch\PYZus{}folder}\PY{p}{)}

    \PY{k}{for} \PY{n}{hidden\PYZus{}layer\PYZus{}neuron\PYZus{}number} \PY{o+ow}{in} \PY{n}{hidden\PYZus{}layer\PYZus{}neuron\PYZus{}numbers}\PY{p}{:}
        \PY{n}{hidden\PYZus{}layer\PYZus{}folder} \PY{o}{=} \PY{l+s+sa}{f}\PY{l+s+s2}{\PYZdq{}}\PY{l+s+si}{\PYZob{}}\PY{n}{batch\PYZus{}folder}\PY{l+s+si}{\PYZcb{}}\PY{l+s+s2}{/hidden\PYZhy{}layer\PYZhy{}neuron\PYZhy{}number\PYZhy{}}\PY{l+s+si}{\PYZob{}}\PY{n}{hidden\PYZus{}layer\PYZus{}neuron\PYZus{}number}\PY{l+s+si}{\PYZcb{}}\PY{l+s+s2}{\PYZdq{}}
        \PY{k}{if} \PY{o+ow}{not} \PY{n}{os}\PY{o}{.}\PY{n}{path}\PY{o}{.}\PY{n}{exists}\PY{p}{(}\PY{n}{hidden\PYZus{}layer\PYZus{}folder}\PY{p}{)}\PY{p}{:}
            \PY{n}{os}\PY{o}{.}\PY{n}{makedirs}\PY{p}{(}\PY{n}{hidden\PYZus{}layer\PYZus{}folder}\PY{p}{)}

        \PY{k}{for} \PY{n}{learning\PYZus{}rate} \PY{o+ow}{in} \PY{n}{learning\PYZus{}rates}\PY{p}{:}
            \PY{n}{mlp}\PY{p}{,} \PY{n}{history} \PY{o}{=} \PY{n}{trainMLP}\PY{p}{(}\PY{n}{metric}\PY{p}{,} \PY{n}{epochs}\PY{p}{,} \PY{n}{hidden\PYZus{}layer\PYZus{}neuron\PYZus{}number}\PY{p}{,} \PY{n}{learning\PYZus{}rate}\PY{p}{,} \PY{n}{batch\PYZus{}size}\PY{p}{,} \PY{n}{test\PYZus{}set\PYZus{}percentage}\PY{p}{)}
            \PY{n}{saveMLP}\PY{p}{(}\PY{n}{hidden\PYZus{}layer\PYZus{}folder}\PY{p}{,} \PY{n}{mlp}\PY{p}{,} \PY{n}{learning\PYZus{}rate}\PY{p}{,} \PY{n}{history}\PY{p}{)}
\end{Verbatim}
\end{tcolorbox}

    Carregando os modelos treinados para a memória

    \begin{tcolorbox}[breakable, size=fbox, boxrule=1pt, pad at break*=1mm,colback=cellbackground, colframe=cellborder]
\prompt{In}{incolor}{2}{\boxspacing}
\begin{Verbatim}[commandchars=\\\{\}]
\PY{n}{params} \PY{o}{=} \PY{n}{json}\PY{o}{.}\PY{n}{load}\PY{p}{(}\PY{n+nb}{open}\PY{p}{(}\PY{l+s+s1}{\PYZsq{}}\PY{l+s+s1}{params.json}\PY{l+s+s1}{\PYZsq{}}\PY{p}{)}\PY{p}{)}

\PY{n}{test\PYZus{}set\PYZus{}percentage} \PY{o}{=} \PY{n}{params}\PY{p}{[}\PY{l+s+s1}{\PYZsq{}}\PY{l+s+s1}{test\PYZus{}set\PYZus{}percentage}\PY{l+s+s1}{\PYZsq{}}\PY{p}{]}
\PY{n}{metric} \PY{o}{=} \PY{n}{params}\PY{p}{[}\PY{l+s+s1}{\PYZsq{}}\PY{l+s+s1}{metric}\PY{l+s+s1}{\PYZsq{}}\PY{p}{]}
\PY{n}{epochs} \PY{o}{=} \PY{n}{params}\PY{p}{[}\PY{l+s+s1}{\PYZsq{}}\PY{l+s+s1}{epochs}\PY{l+s+s1}{\PYZsq{}}\PY{p}{]}

\PY{c+c1}{\PYZsh{} Ordenado de forma que as primeiras execuções sejam mais rápidas}
\PY{n}{batch\PYZus{}sizes} \PY{o}{=} \PY{p}{[}\PY{n+nb}{round}\PY{p}{(}\PY{l+m+mi}{5000} \PY{o}{*} \PY{n+nb}{round}\PY{p}{(}\PY{l+m+mi}{1} \PY{o}{\PYZhy{}} \PY{n}{params}\PY{p}{[}\PY{l+s+s1}{\PYZsq{}}\PY{l+s+s1}{test\PYZus{}set\PYZus{}percentage}\PY{l+s+s1}{\PYZsq{}}\PY{p}{]}\PY{p}{,} \PY{l+m+mi}{2}\PY{p}{)}\PY{p}{)}\PY{p}{]}
\PY{k}{for} \PY{n}{batch\PYZus{}size} \PY{o+ow}{in} \PY{n}{params}\PY{o}{.}\PY{n}{get}\PY{p}{(}\PY{l+s+s1}{\PYZsq{}}\PY{l+s+s1}{batch\PYZus{}sizes}\PY{l+s+s1}{\PYZsq{}}\PY{p}{)}\PY{p}{:}
    \PY{n}{batch\PYZus{}sizes}\PY{o}{.}\PY{n}{append}\PY{p}{(}\PY{n}{batch\PYZus{}size}\PY{p}{)}

\PY{n}{hidden\PYZus{}layer\PYZus{}neuron\PYZus{}numbers} \PY{o}{=} \PY{n}{params}\PY{p}{[}\PY{l+s+s1}{\PYZsq{}}\PY{l+s+s1}{hidden\PYZus{}layer\PYZus{}neuron\PYZus{}numbers}\PY{l+s+s1}{\PYZsq{}}\PY{p}{]}
\PY{n}{learning\PYZus{}rates} \PY{o}{=} \PY{n}{params}\PY{p}{[}\PY{l+s+s1}{\PYZsq{}}\PY{l+s+s1}{learning\PYZus{}rates}\PY{l+s+s1}{\PYZsq{}}\PY{p}{]}

\PY{n}{models} \PY{o}{=} \PY{p}{\PYZob{}}\PY{p}{\PYZcb{}}

\PY{k}{for} \PY{n}{i}\PY{p}{,} \PY{n}{batch\PYZus{}size} \PY{o+ow}{in} \PY{n+nb}{enumerate}\PY{p}{(}\PY{n}{batch\PYZus{}sizes}\PY{p}{)}\PY{p}{:}
    \PY{n}{batch\PYZus{}folder} \PY{o}{=} \PY{l+s+sa}{f}\PY{l+s+s2}{\PYZdq{}}\PY{l+s+s2}{models/batch\PYZhy{}size\PYZhy{}}\PY{l+s+si}{\PYZob{}}\PY{n}{batch\PYZus{}size}\PY{l+s+si}{\PYZcb{}}\PY{l+s+s2}{\PYZdq{}}
    \PY{k}{if} \PY{o+ow}{not} \PY{n}{os}\PY{o}{.}\PY{n}{path}\PY{o}{.}\PY{n}{exists}\PY{p}{(}\PY{n}{batch\PYZus{}folder}\PY{p}{)}\PY{p}{:}
        \PY{n}{os}\PY{o}{.}\PY{n}{makedirs}\PY{p}{(}\PY{n}{batch\PYZus{}folder}\PY{p}{)}

    \PY{k}{for} \PY{n}{j}\PY{p}{,} \PY{n}{hidden\PYZus{}layer\PYZus{}neuron\PYZus{}number} \PY{o+ow}{in} \PY{n+nb}{enumerate}\PY{p}{(}\PY{n}{hidden\PYZus{}layer\PYZus{}neuron\PYZus{}numbers}\PY{p}{)}\PY{p}{:}
        \PY{n}{hidden\PYZus{}layer\PYZus{}folder} \PY{o}{=} \PY{l+s+sa}{f}\PY{l+s+s2}{\PYZdq{}}\PY{l+s+si}{\PYZob{}}\PY{n}{batch\PYZus{}folder}\PY{l+s+si}{\PYZcb{}}\PY{l+s+s2}{/hidden\PYZhy{}layer\PYZhy{}neuron\PYZhy{}number\PYZhy{}}\PY{l+s+si}{\PYZob{}}\PY{n}{hidden\PYZus{}layer\PYZus{}neuron\PYZus{}number}\PY{l+s+si}{\PYZcb{}}\PY{l+s+s2}{\PYZdq{}}
        
        \PY{k}{for} \PY{n}{k}\PY{p}{,} \PY{n}{learning\PYZus{}rate} \PY{o+ow}{in} \PY{n+nb}{enumerate}\PY{p}{(}\PY{n}{learning\PYZus{}rates}\PY{p}{)}\PY{p}{:}
            \PY{n}{mlp} \PY{o}{=} \PY{n}{load\PYZus{}model}\PY{p}{(}\PY{l+s+sa}{f}\PY{l+s+s2}{\PYZdq{}}\PY{l+s+si}{\PYZob{}}\PY{n}{hidden\PYZus{}layer\PYZus{}folder}\PY{l+s+si}{\PYZcb{}}\PY{l+s+s2}{/mlp\PYZhy{}lr}\PY{l+s+si}{\PYZob{}}\PY{n}{learning\PYZus{}rate}\PY{l+s+si}{\PYZcb{}}\PY{l+s+s2}{\PYZdq{}}\PY{p}{)}
            \PY{n}{history} \PY{o}{=} \PY{n}{pickle}\PY{o}{.}\PY{n}{load}\PY{p}{(}\PY{n+nb}{open}\PY{p}{(}\PY{l+s+sa}{f}\PY{l+s+s2}{\PYZdq{}}\PY{l+s+si}{\PYZob{}}\PY{n}{hidden\PYZus{}layer\PYZus{}folder}\PY{l+s+si}{\PYZcb{}}\PY{l+s+s2}{/history\PYZhy{}lr}\PY{l+s+si}{\PYZob{}}\PY{n}{learning\PYZus{}rate}\PY{l+s+si}{\PYZcb{}}\PY{l+s+s2}{.pkl}\PY{l+s+s2}{\PYZdq{}}\PY{p}{,} \PY{l+s+s1}{\PYZsq{}}\PY{l+s+s1}{rb}\PY{l+s+s1}{\PYZsq{}}\PY{p}{)}\PY{p}{)}

            \PY{k}{if} \PY{n}{models}\PY{o}{.}\PY{n}{get}\PY{p}{(}\PY{n}{hidden\PYZus{}layer\PYZus{}neuron\PYZus{}number}\PY{p}{)} \PY{o+ow}{is} \PY{k+kc}{None}\PY{p}{:}
                \PY{n}{models}\PY{p}{[}\PY{n}{hidden\PYZus{}layer\PYZus{}neuron\PYZus{}number}\PY{p}{]} \PY{o}{=} \PY{p}{\PYZob{}}\PY{p}{\PYZcb{}}

            \PY{k}{if} \PY{n}{models}\PY{p}{[}\PY{n}{hidden\PYZus{}layer\PYZus{}neuron\PYZus{}number}\PY{p}{]}\PY{o}{.}\PY{n}{get}\PY{p}{(}\PY{n}{batch\PYZus{}size}\PY{p}{)} \PY{o+ow}{is} \PY{k+kc}{None}\PY{p}{:}
                \PY{n}{models}\PY{p}{[}\PY{n}{hidden\PYZus{}layer\PYZus{}neuron\PYZus{}number}\PY{p}{]}\PY{p}{[}\PY{n}{batch\PYZus{}size}\PY{p}{]} \PY{o}{=} \PY{p}{\PYZob{}}\PY{p}{\PYZcb{}}
            
            \PY{n}{models}\PY{p}{[}\PY{n}{hidden\PYZus{}layer\PYZus{}neuron\PYZus{}number}\PY{p}{]}\PY{p}{[}\PY{n}{batch\PYZus{}size}\PY{p}{]}\PY{p}{[}\PY{n}{learning\PYZus{}rate}\PY{p}{]} \PY{o}{=} \PY{p}{(}\PY{n}{mlp}\PY{p}{,} \PY{n}{history}\PY{p}{)}
\end{Verbatim}
\end{tcolorbox}

    \begin{Verbatim}[commandchars=\\\{\}]
2023-05-24 22:24:45.928955: I
tensorflow/compiler/xla/stream\_executor/cuda/cuda\_gpu\_executor.cc:982] could not
open file to read NUMA node: /sys/bus/pci/devices/0000:01:00.0/numa\_node
Your kernel may have been built without NUMA support.
2023-05-24 22:24:45.929412: W
tensorflow/core/common\_runtime/gpu/gpu\_device.cc:1956] Cannot dlopen some GPU
libraries. Please make sure the missing libraries mentioned above are installed
properly if you would like to use GPU. Follow the guide at
https://www.tensorflow.org/install/gpu for how to download and setup the
required libraries for your platform.
Skipping registering GPU devices{\ldots}
    \end{Verbatim}

    Criando a função que irá fazer o display dos gŕaficos de erro empirico
em função da época.

    \begin{tcolorbox}[breakable, size=fbox, boxrule=1pt, pad at break*=1mm,colback=cellbackground, colframe=cellborder]
\prompt{In}{incolor}{3}{\boxspacing}
\begin{Verbatim}[commandchars=\\\{\}]
\PY{k}{def} \PY{n+nf}{displayMLP}\PY{p}{(}\PY{n}{desired\PYZus{}neuron\PYZus{}nb}\PY{p}{)}\PY{p}{:}
    \PY{k}{for} \PY{n}{hidden\PYZus{}layer\PYZus{}neuron\PYZus{}number}\PY{p}{,} \PY{n}{model\PYZus{}group} \PY{o+ow}{in} \PY{n}{models}\PY{o}{.}\PY{n}{items}\PY{p}{(}\PY{p}{)}\PY{p}{:}
        \PY{k}{if} \PY{n}{hidden\PYZus{}layer\PYZus{}neuron\PYZus{}number} \PY{o}{!=} \PY{n}{desired\PYZus{}neuron\PYZus{}nb}\PY{p}{:}
            \PY{k}{continue}
        
        \PY{n}{row} \PY{o}{=} \PY{o}{\PYZhy{}}\PY{l+m+mi}{1}
        \PY{n}{fig}\PY{p}{,} \PY{n}{axs} \PY{o}{=} \PY{n}{plt}\PY{o}{.}\PY{n}{subplots}\PY{p}{(}\PY{l+m+mi}{4}\PY{p}{,} \PY{l+m+mi}{3}\PY{p}{,} \PY{n}{figsize}\PY{o}{=}\PY{p}{(}\PY{l+m+mi}{20}\PY{p}{,}\PY{l+m+mi}{15}\PY{p}{)}\PY{p}{)}
        \PY{k}{for} \PY{n}{batch\PYZus{}size}\PY{p}{,} \PY{n}{model\PYZus{}group} \PY{o+ow}{in} \PY{n}{model\PYZus{}group}\PY{o}{.}\PY{n}{items}\PY{p}{(}\PY{p}{)}\PY{p}{:}
            \PY{n}{colors} \PY{o}{=} \PY{p}{[}\PY{l+s+s1}{\PYZsq{}}\PY{l+s+s1}{blue}\PY{l+s+s1}{\PYZsq{}}\PY{p}{,} \PY{l+s+s1}{\PYZsq{}}\PY{l+s+s1}{green}\PY{l+s+s1}{\PYZsq{}}\PY{p}{,} \PY{l+s+s1}{\PYZsq{}}\PY{l+s+s1}{red}\PY{l+s+s1}{\PYZsq{}}\PY{p}{]}
            
            \PY{n}{color\PYZus{}index} \PY{o}{=} \PY{o}{\PYZhy{}}\PY{l+m+mi}{1}
            \PY{n}{row} \PY{o}{+}\PY{o}{=} \PY{l+m+mi}{1}

            \PY{c+c1}{\PYZsh{} fig, axs = plt.subplots(3, len(model\PYZus{}group), figsize=(15,5))}
            \PY{n}{counter} \PY{o}{=} \PY{o}{\PYZhy{}}\PY{l+m+mi}{1}
            \PY{k}{for} \PY{n}{i}\PY{p}{,} \PY{p}{(}\PY{n}{learning\PYZus{}rate}\PY{p}{,} \PY{n}{model}\PY{p}{)} \PY{o+ow}{in} \PY{n+nb}{enumerate}\PY{p}{(}\PY{n}{model\PYZus{}group}\PY{o}{.}\PY{n}{items}\PY{p}{(}\PY{p}{)}\PY{p}{)}\PY{p}{:}
                \PY{n}{labels} \PY{o}{=} \PY{p}{[}\PY{p}{]}

                \PY{n}{color\PYZus{}index} \PY{o}{+}\PY{o}{=} \PY{l+m+mi}{1}
                \PY{n}{counter} \PY{o}{+}\PY{o}{=} \PY{l+m+mi}{1}
                \PY{n}{color} \PY{o}{=} \PY{n}{colors}\PY{p}{[}\PY{n}{color\PYZus{}index}\PY{p}{]}
                \PY{n}{mlp}\PY{p}{,} \PY{n}{history} \PY{o}{=} \PY{n}{model}

                \PY{c+c1}{\PYZsh{} axs[row, i].plot(history[\PYZsq{}val\PYZus{}loss\PYZsq{}], color=color, linestyle=\PYZsq{}dashed\PYZsq{}, alpha=0.3)}
                \PY{n}{axs}\PY{p}{[}\PY{n}{row}\PY{p}{,} \PY{n}{i}\PY{p}{]}\PY{o}{.}\PY{n}{plot}\PY{p}{(}\PY{n}{history}\PY{p}{[}\PY{l+s+s1}{\PYZsq{}}\PY{l+s+s1}{loss}\PY{l+s+s1}{\PYZsq{}}\PY{p}{]}\PY{p}{,} \PY{n}{color}\PY{o}{=}\PY{n}{color}\PY{p}{,} \PY{n}{linestyle}\PY{o}{=}\PY{l+s+s1}{\PYZsq{}}\PY{l+s+s1}{solid}\PY{l+s+s1}{\PYZsq{}}\PY{p}{)}
                
                \PY{c+c1}{\PYZsh{} labels.append(f\PYZdq{}Teste (LR=\PYZob{}learning\PYZus{}rate\PYZcb{})\PYZdq{})}
                \PY{n}{labels}\PY{o}{.}\PY{n}{append}\PY{p}{(}\PY{l+s+sa}{f}\PY{l+s+s2}{\PYZdq{}}\PY{l+s+s2}{LR=}\PY{l+s+si}{\PYZob{}}\PY{n}{learning\PYZus{}rate}\PY{l+s+si}{\PYZcb{}}\PY{l+s+s2}{)}\PY{l+s+s2}{\PYZdq{}}\PY{p}{)}

                \PY{n}{axs}\PY{p}{[}\PY{n}{row}\PY{p}{,} \PY{n}{i}\PY{p}{]}\PY{o}{.}\PY{n}{set\PYZus{}ylabel}\PY{p}{(}\PY{l+s+s1}{\PYZsq{}}\PY{l+s+s1}{Erro empírico}\PY{l+s+s1}{\PYZsq{}}\PY{p}{)}
                \PY{n}{axs}\PY{p}{[}\PY{n}{row}\PY{p}{,} \PY{n}{i}\PY{p}{]}\PY{o}{.}\PY{n}{set\PYZus{}xlabel}\PY{p}{(}\PY{l+s+s1}{\PYZsq{}}\PY{l+s+s1}{Época}\PY{l+s+s1}{\PYZsq{}}\PY{p}{)}
                \PY{n}{axs}\PY{p}{[}\PY{n}{row}\PY{p}{,} \PY{n}{i}\PY{p}{]}\PY{o}{.}\PY{n}{legend}\PY{p}{(}\PY{n}{labels}\PY{p}{,} \PY{n}{loc}\PY{o}{=}\PY{l+s+s1}{\PYZsq{}}\PY{l+s+s1}{upper right}\PY{l+s+s1}{\PYZsq{}}\PY{p}{)}

                \PY{k}{if} \PY{n}{batch\PYZus{}size} \PY{o}{==} \PY{l+m+mi}{3750}\PY{p}{:}
                    \PY{n}{batch\PYZus{}size} \PY{o}{=} \PY{l+s+s2}{\PYZdq{}}\PY{l+s+s2}{GD}\PY{l+s+s2}{\PYZdq{}}

                \PY{k}{if} \PY{n}{batch\PYZus{}size} \PY{o}{==} \PY{l+m+mi}{1}\PY{p}{:}
                    \PY{n}{batch\PYZus{}size} \PY{o}{=} \PY{l+s+s2}{\PYZdq{}}\PY{l+s+s2}{SGD}\PY{l+s+s2}{\PYZdq{}}

                \PY{n}{axs}\PY{p}{[}\PY{n}{row}\PY{p}{,} \PY{l+m+mi}{1}\PY{p}{]}\PY{o}{.}\PY{n}{set\PYZus{}title}\PY{p}{(}\PY{l+s+sa}{f}\PY{l+s+s2}{\PYZdq{}}\PY{l+s+s2}{Batch size: }\PY{l+s+si}{\PYZob{}}\PY{n}{batch\PYZus{}size}\PY{l+s+si}{\PYZcb{}}\PY{l+s+s2}{\PYZdq{}}\PY{p}{,} \PY{n}{fontsize}\PY{o}{=}\PY{l+m+mi}{16}\PY{p}{,} \PY{n}{fontweight}\PY{o}{=}\PY{l+s+s1}{\PYZsq{}}\PY{l+s+s1}{bold}\PY{l+s+s1}{\PYZsq{}}\PY{p}{)}

            \PY{n}{plt}\PY{o}{.}\PY{n}{suptitle}\PY{p}{(}\PY{l+s+sa}{f}\PY{l+s+s2}{\PYZdq{}}\PY{l+s+s2}{Neurônios na camada oculta: }\PY{l+s+si}{\PYZob{}}\PY{n}{hidden\PYZus{}layer\PYZus{}neuron\PYZus{}number}\PY{l+s+si}{\PYZcb{}}\PY{l+s+s2}{\PYZdq{}}\PY{p}{,} \PY{n}{fontsize}\PY{o}{=}\PY{l+m+mi}{20}\PY{p}{,} \PY{n}{fontweight}\PY{o}{=}\PY{l+s+s1}{\PYZsq{}}\PY{l+s+s1}{bold}\PY{l+s+s1}{\PYZsq{}}\PY{p}{)}

        \PY{n}{plt}\PY{o}{.}\PY{n}{subplots\PYZus{}adjust}\PY{p}{(}\PY{n}{hspace}\PY{o}{=}\PY{l+m+mf}{0.5}\PY{p}{)}
        \PY{c+c1}{\PYZsh{} draw gridlines}
        \PY{k}{for} \PY{n}{ax} \PY{o+ow}{in} \PY{n}{axs}\PY{o}{.}\PY{n}{flat}\PY{p}{:}
            \PY{n}{ax}\PY{o}{.}\PY{n}{grid}\PY{p}{(}\PY{k+kc}{True}\PY{p}{)}
            
        \PY{n}{plt}\PY{o}{.}\PY{n}{show}\PY{p}{(}\PY{p}{)}
        \PY{n}{plt}\PY{o}{.}\PY{n}{clf}\PY{p}{(}\PY{p}{)}
\end{Verbatim}
\end{tcolorbox}

    Com relação ao learning rate, podemos observar que quanto maior ele é,
mais instável o algoritmo se torna, tendo dificuldades em convergir para
algum valor de erro. Isso pode ser devido ao fato de que ele está sempre
``ultrapassando'' o mínimo encontrado pelo modelo, como se estivesse
andando pela região próxima a ele em vez de chegar cada vez mais perto.
Quanto maior o valor do learning rate, maior cada passo da descida de
gradiente se torna, aumentando a chance de ultrapassar o mínimo e
dificultando a convergência do algoritmo.

Focando nos modelos com learning\_rate=10, os com batch\_size=GD ou 50
exibem esse comportamento com a maior margem de variação. Já os com
batch\_size=10 ou 1 possuem menos variação, talvez devido à topologia do
mínimo local encontrado. Os modelos com learning\_rate=1 ou 0.5 são
semelhantes e possuem menos variação que os de 10.

O tamanho do batch afeta principalmente para qual número o erro empírico
está convergindo. Com batches menores, o erro converge para valores
maiores. Quando diminuímos o batch size, o caminho da descida de
gradiente se torna mais caótico (como é evidente no SGD), aumentando a
chance de desviar para uma região diferente da desejada (mínimo global).
Uma vantagem de ter um batch size grande é que o algoritmo se torna
menos suscetível às variações individuais dos dados e calcula com mais
precisão a direção que minimiza o erro global. Com um batch pequeno, uma
grande variação em um único dado afeta muito a direção resultante da
descida.

Focando nos modelos com learning\_rate=1 ou 0.5, os com batch\_size=GD
se mostram especialmente bons, enquanto os demais parecem enfrentar
problemas com mínimos locais.

    \begin{tcolorbox}[breakable, size=fbox, boxrule=1pt, pad at break*=1mm,colback=cellbackground, colframe=cellborder]
\prompt{In}{incolor}{4}{\boxspacing}
\begin{Verbatim}[commandchars=\\\{\}]
\PY{n}{displayMLP}\PY{p}{(}\PY{l+m+mi}{25}\PY{p}{)}
\end{Verbatim}
\end{tcolorbox}

    \begin{center}
    \adjustimage{max size={0.9\linewidth}{0.9\paperheight}}{output_9_0.png}
    \end{center}
    { \hspace*{\fill} \\}
    
    
    \begin{Verbatim}[commandchars=\\\{\}]
<Figure size 640x480 with 0 Axes>
    \end{Verbatim}

    
    \begin{tcolorbox}[breakable, size=fbox, boxrule=1pt, pad at break*=1mm,colback=cellbackground, colframe=cellborder]
\prompt{In}{incolor}{5}{\boxspacing}
\begin{Verbatim}[commandchars=\\\{\}]
\PY{n}{displayMLP}\PY{p}{(}\PY{l+m+mi}{50}\PY{p}{)}
\end{Verbatim}
\end{tcolorbox}

    \begin{center}
    \adjustimage{max size={0.9\linewidth}{0.9\paperheight}}{output_10_0.png}
    \end{center}
    { \hspace*{\fill} \\}
    
    
    \begin{Verbatim}[commandchars=\\\{\}]
<Figure size 640x480 with 0 Axes>
    \end{Verbatim}

    
    \begin{tcolorbox}[breakable, size=fbox, boxrule=1pt, pad at break*=1mm,colback=cellbackground, colframe=cellborder]
\prompt{In}{incolor}{6}{\boxspacing}
\begin{Verbatim}[commandchars=\\\{\}]
\PY{n}{displayMLP}\PY{p}{(}\PY{l+m+mi}{100}\PY{p}{)}
\end{Verbatim}
\end{tcolorbox}

    \begin{center}
    \adjustimage{max size={0.9\linewidth}{0.9\paperheight}}{output_11_0.png}
    \end{center}
    { \hspace*{\fill} \\}
    
    
    \begin{Verbatim}[commandchars=\\\{\}]
<Figure size 640x480 with 0 Axes>
    \end{Verbatim}

    
    Função para plotar a variação do número de unidades na camada oculta
para cada um dos três algoritmos de cálculo de gradiente.

    \begin{tcolorbox}[breakable, size=fbox, boxrule=1pt, pad at break*=1mm,colback=cellbackground, colframe=cellborder]
\prompt{In}{incolor}{7}{\boxspacing}
\begin{Verbatim}[commandchars=\\\{\}]
\PY{k}{def} \PY{n+nf}{displayNeuronErrors}\PY{p}{(}\PY{p}{)}\PY{p}{:}
    \PY{n}{empirical\PYZus{}data} \PY{o}{=} \PY{p}{\PYZob{}}\PY{p}{\PYZcb{}}
    \PY{k}{for} \PY{n}{hidden\PYZus{}layer\PYZus{}neuron\PYZus{}number}\PY{p}{,} \PY{n}{model\PYZus{}group} \PY{o+ow}{in} \PY{n}{models}\PY{o}{.}\PY{n}{items}\PY{p}{(}\PY{p}{)}\PY{p}{:}
        \PY{n}{index} \PY{o}{=} \PY{o}{\PYZhy{}}\PY{l+m+mi}{1} 

        \PY{k}{for} \PY{n}{batch\PYZus{}size}\PY{p}{,} \PY{n}{model\PYZus{}group} \PY{o+ow}{in} \PY{n}{model\PYZus{}group}\PY{o}{.}\PY{n}{items}\PY{p}{(}\PY{p}{)}\PY{p}{:}
            \PY{n}{index} \PY{o}{+}\PY{o}{=} \PY{l+m+mi}{1}
            \PY{n}{last\PYZus{}epoch\PYZus{}loss} \PY{o}{=} \PY{k+kc}{None}
            \PY{n}{last\PYZus{}epoch\PYZus{}eval\PYZus{}loss} \PY{o}{=} \PY{k+kc}{None}

            \PY{k}{for} \PY{n}{learning\PYZus{}rate}\PY{p}{,} \PY{n}{model} \PY{o+ow}{in} \PY{n}{model\PYZus{}group}\PY{o}{.}\PY{n}{items}\PY{p}{(}\PY{p}{)}\PY{p}{:}
                \PY{k}{if} \PY{n}{learning\PYZus{}rate} \PY{o}{==} \PY{l+m+mf}{0.5}\PY{p}{:}
                    \PY{n}{mlp}\PY{p}{,} \PY{n}{history} \PY{o}{=} \PY{n}{model}
                    \PY{n}{last\PYZus{}epoch\PYZus{}loss} \PY{o}{=} \PY{n}{history}\PY{p}{[}\PY{l+s+s1}{\PYZsq{}}\PY{l+s+s1}{loss}\PY{l+s+s1}{\PYZsq{}}\PY{p}{]}\PY{p}{[}\PY{o}{\PYZhy{}}\PY{l+m+mi}{1}\PY{p}{]}
                    \PY{n}{last\PYZus{}epoch\PYZus{}eval\PYZus{}loss} \PY{o}{=} \PY{n}{history}\PY{p}{[}\PY{l+s+s1}{\PYZsq{}}\PY{l+s+s1}{val\PYZus{}loss}\PY{l+s+s1}{\PYZsq{}}\PY{p}{]}\PY{p}{[}\PY{o}{\PYZhy{}}\PY{l+m+mi}{1}\PY{p}{]}

            \PY{n}{empirical\PYZus{}data}\PY{p}{[}\PY{n}{hidden\PYZus{}layer\PYZus{}neuron\PYZus{}number}\PY{p}{]} \PY{o}{=} \PY{n}{empirical\PYZus{}data}\PY{o}{.}\PY{n}{get}\PY{p}{(}\PY{n}{hidden\PYZus{}layer\PYZus{}neuron\PYZus{}number}\PY{p}{,} \PY{p}{[}\PY{p}{]}\PY{p}{)}
            \PY{n}{empirical\PYZus{}data}\PY{p}{[}\PY{n}{hidden\PYZus{}layer\PYZus{}neuron\PYZus{}number}\PY{p}{]}\PY{o}{.}\PY{n}{append}\PY{p}{(}\PY{n}{last\PYZus{}epoch\PYZus{}loss}\PY{p}{)}

    \PY{n}{x} \PY{o}{=} \PY{n}{params}\PY{p}{[}\PY{l+s+s1}{\PYZsq{}}\PY{l+s+s1}{hidden\PYZus{}layer\PYZus{}neuron\PYZus{}numbers}\PY{l+s+s1}{\PYZsq{}}\PY{p}{]}
    \PY{n}{colors} \PY{o}{=} \PY{p}{[}\PY{l+s+s1}{\PYZsq{}}\PY{l+s+s1}{blue}\PY{l+s+s1}{\PYZsq{}}\PY{p}{,} \PY{l+s+s1}{\PYZsq{}}\PY{l+s+s1}{green}\PY{l+s+s1}{\PYZsq{}}\PY{p}{,} \PY{l+s+s1}{\PYZsq{}}\PY{l+s+s1}{red}\PY{l+s+s1}{\PYZsq{}}\PY{p}{,} \PY{l+s+s1}{\PYZsq{}}\PY{l+s+s1}{purple}\PY{l+s+s1}{\PYZsq{}}\PY{p}{]}
    \PY{k}{for} \PY{n}{i} \PY{o+ow}{in} \PY{n+nb}{range}\PY{p}{(}\PY{n+nb}{len}\PY{p}{(}\PY{n}{batch\PYZus{}sizes}\PY{p}{)}\PY{p}{)}\PY{p}{:}
        \PY{n}{empirical\PYZus{}y} \PY{o}{=} \PY{p}{[}\PY{p}{]}

        \PY{k}{for} \PY{n}{neuron\PYZus{}nb}\PY{p}{,} \PY{n}{batch\PYZus{}errors} \PY{o+ow}{in} \PY{n}{empirical\PYZus{}data}\PY{o}{.}\PY{n}{items}\PY{p}{(}\PY{p}{)}\PY{p}{:}
            \PY{n}{empirical\PYZus{}y}\PY{o}{.}\PY{n}{append}\PY{p}{(}\PY{n}{batch\PYZus{}errors}\PY{p}{[}\PY{n}{i}\PY{p}{]}\PY{p}{)}

        \PY{n}{batch\PYZus{}size} \PY{o}{=} \PY{n}{batch\PYZus{}sizes}\PY{p}{[}\PY{n}{i}\PY{p}{]}
        \PY{k}{if} \PY{n}{batch\PYZus{}size} \PY{o}{==} \PY{l+m+mi}{3750}\PY{p}{:}
            \PY{n}{batch\PYZus{}size} \PY{o}{=} \PY{l+s+s2}{\PYZdq{}}\PY{l+s+s2}{GD}\PY{l+s+s2}{\PYZdq{}}

        \PY{k}{if} \PY{n}{batch\PYZus{}size} \PY{o}{==} \PY{l+m+mi}{1}\PY{p}{:}
            \PY{n}{batch\PYZus{}size} \PY{o}{=} \PY{l+s+s2}{\PYZdq{}}\PY{l+s+s2}{SGD}\PY{l+s+s2}{\PYZdq{}}

        \PY{n}{plt}\PY{o}{.}\PY{n}{plot}\PY{p}{(}\PY{n}{x}\PY{p}{,} \PY{n}{empirical\PYZus{}y}\PY{p}{,} \PY{n}{label}\PY{o}{=}\PY{l+s+sa}{f}\PY{l+s+s2}{\PYZdq{}}\PY{l+s+s2}{Batch size: }\PY{l+s+si}{\PYZob{}}\PY{n}{batch\PYZus{}size}\PY{l+s+si}{\PYZcb{}}\PY{l+s+s2}{\PYZdq{}}\PY{p}{,} \PY{n}{color}\PY{o}{=}\PY{n}{colors}\PY{p}{[}\PY{n}{i}\PY{p}{]}\PY{p}{)}

    \PY{c+c1}{\PYZsh{} invert the order of the labels}
    \PY{n}{handles}\PY{p}{,} \PY{n}{labels} \PY{o}{=} \PY{n}{plt}\PY{o}{.}\PY{n}{gca}\PY{p}{(}\PY{p}{)}\PY{o}{.}\PY{n}{get\PYZus{}legend\PYZus{}handles\PYZus{}labels}\PY{p}{(}\PY{p}{)}
    \PY{n}{plt}\PY{o}{.}\PY{n}{legend}\PY{p}{(}\PY{n}{handles}\PY{p}{[}\PY{p}{:}\PY{p}{:}\PY{o}{\PYZhy{}}\PY{l+m+mi}{1}\PY{p}{]}\PY{p}{,} \PY{n}{labels}\PY{p}{[}\PY{p}{:}\PY{p}{:}\PY{o}{\PYZhy{}}\PY{l+m+mi}{1}\PY{p}{]}\PY{p}{,} \PY{n}{loc}\PY{o}{=}\PY{l+s+s1}{\PYZsq{}}\PY{l+s+s1}{upper right}\PY{l+s+s1}{\PYZsq{}}\PY{p}{)}

    \PY{n}{plt}\PY{o}{.}\PY{n}{xlabel}\PY{p}{(}\PY{l+s+s1}{\PYZsq{}}\PY{l+s+s1}{Número de neurônios na camada oculta}\PY{l+s+s1}{\PYZsq{}}\PY{p}{)}
    \PY{n}{plt}\PY{o}{.}\PY{n}{ylabel}\PY{p}{(}\PY{l+s+s1}{\PYZsq{}}\PY{l+s+s1}{Erro impírico na última época}\PY{l+s+s1}{\PYZsq{}}\PY{p}{)}
    \PY{c+c1}{\PYZsh{} draw gridlines}
    \PY{n}{plt}\PY{o}{.}\PY{n}{grid}\PY{p}{(}\PY{k+kc}{True}\PY{p}{)}
    \PY{n}{plt}\PY{o}{.}\PY{n}{xticks}\PY{p}{(}\PY{n}{x}\PY{p}{)}
    \PY{n}{plt}\PY{o}{.}\PY{n}{show}\PY{p}{(}\PY{p}{)}
\end{Verbatim}
\end{tcolorbox}

    Para simplificar a análise, escolhi os modelos com o menor learning rate
(0.5), pois os resultados obtidos com outros learning rates são piores.

Podemos observar pelo gráfico abaixo que quanto maior o batch\_size,
menor o valor de convergência do erro empírico. Isso se deve à natureza
caótica resultante de se ter batches menores.

Ao aumentar o número de neurônios na camada oculta, estamos aumentando a
capacidade do modelo. É esperado que o erro empírico de modelos com
maior capacidade seja menor, pois eles podem gerar funções mais
complexas. Esse fato é evidenciado no modelo com batch\_size=GD, mas não
nos demais. A explicação para isso é que esses modelos acabaram caindo
em mínimos locais devido aos menores batch\_sizes. É fácil verificar
essa hipótese ao retornar para os gráficos acima e observar como esses
modelos caem em vários mínimos locais ao longo das épocas (até na
última).

    \begin{tcolorbox}[breakable, size=fbox, boxrule=1pt, pad at break*=1mm,colback=cellbackground, colframe=cellborder]
\prompt{In}{incolor}{8}{\boxspacing}
\begin{Verbatim}[commandchars=\\\{\}]
\PY{n}{displayNeuronErrors}\PY{p}{(}\PY{p}{)}
\end{Verbatim}
\end{tcolorbox}

    \begin{center}
    \adjustimage{max size={0.9\linewidth}{0.9\paperheight}}{output_15_0.png}
    \end{center}
    { \hspace*{\fill} \\}
    

    % Add a bibliography block to the postdoc
    
    
    
\end{document}
